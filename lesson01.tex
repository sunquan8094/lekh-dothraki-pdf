\chapter{Personal Pronouns and Present Tense of Verbs}
\section*{Vocabulary}
\begin{itemize}
	\item \textbf{anha} [\textit{pn}]: I
	\item \textbf{yer} [\textit{pn}]: you (singular)
	\item \textbf{me} [\textit{pn}]: he, she, it
	\item \textbf{kisha} [\textit{pn}]: we
	\item \textbf{yeri} [\textit{pn}]: you (plural)
	\item \textbf{mori} [\textit{pn}]: they
	\item \textbf{thirat} [\textit{v-at}]: to live
	\item \textbf{nesolat} [\textit{v-lat}]: to learn
\end{itemize}
\section*{Text}
	\textbf{Anha thirak. Yer thiri. Me thira.} \\
  \textbf{Kisha thiraki. Yeri thiri. Mori thiri.} \\
	\textbf{Anha nesok. Yer nesoe. Me nesoe.} \\
	\textbf{Kisha nesoki. Yeri nesoe. Mori nesoe.}
\section*{Grammar}
You might believe that neither of the words \textbf{thirat} and \textbf{nesolat}
were used at all in the text. They actually were, they were just conjugated.
Like in many Romance langauges, verbs in Dothraki must be conjugated according
to its subject. The two verbs are in the infinitive form in the Vocabulary section.\\
You might also notice that both \textbf{thirat} and \textbf{nesolat} are
conjugated differently. That's because Dothraki has two verb types: \textbf{-at}
verbs and \textbf{-lat} verbs, named for what their infinitive forms end with.
It must be worth noting that while most verbs that end in \textbf{-lat} are \textbf{-lat} verbs,
this is not always the case.
\section*{Exercises}
