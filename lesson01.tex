\chapter{Personal Pronouns and Affirmative Present Tense of Verbs}
\section*{Vocabulary}
\begin{itemize}
	\item \textbf{anha} [\textit{pn}]: I
	\item \textbf{yer} [\textit{pn}]: you (singular)
	\item \textbf{me} [\textit{pn}]: he, she, it
	\item \textbf{kisha} [\textit{pn}]: we
	\item \textbf{yeri} [\textit{pn}]: you (plural)
	\item \textbf{mori} [\textit{pn}]: they
	\item \textbf{thirat} [\textit{v-at}]: to live
	\item \textbf{nesolat} [\textit{v-lat}]: to learn
\end{itemize}
\section*{Text}
	\textbf{Anha thirak. Yer thiri. Me thira.} \\
  \textbf{Kisha thiraki. Yeri thiri. Mori thiri.} \\
	\textbf{Anha nesok. Yer nesoe. Me nesoe.} \\
	\textbf{Kisha nesoki. Yeri nesoe. Mori nesoe.}
\section*{Grammar}
\subsection*{Verb Conjugation}
You might believe that neither of the words \textbf{thirat} and \textbf{nesolat}
were used at all in the text. They actually were, they were just conjugated.
Like in many Romance langauges, verbs in Dothraki must be conjugated according
to its subject. The two verbs are in the infinitive form in the Vocabulary section.\\
You might also notice that both \textbf{thirat} and \textbf{nesolat} are
conjugated differently. That's because Dothraki has two verb types: \textbf{-at}
verbs and \textbf{-lat} verbs, named for what their infinitive forms end with.
It must be worth noting that while most verbs that end in \textbf{-lat} are \textbf{-lat} verbs,
that is not always the case. Always check to see what type of verb it is.\\
\subsection*{Affirmative and Negative}
If a verb is in the present tense, it means that the subject does or does not do the specified action in the present.
Affirmative just talks about the ``does'' part, while the negative talks about the ``does not do'' part. For instance,
take the following sentences:
\begin{itemize}
	\item I think so.
	\item I don't think so.
\end{itemize}
Needless to say, the two sentences are mere opposites of each other because of the fact that the first sentence is in
the affirmative present tense, while the second is in the negative present tense. We'll discuss the negative present
tense in a later lesson.
\subsection*{Conjugation Table}
Below is a table specifying how these verbs should be conjugated in the affirmative present tense. \\
\begin{tabular}{|c|c|c|c|c|}
		\hline
		Pronoun & \textbf{-at} & \textbf{ifat} & \textbf{-lat} & \textbf{dothralat} \\
		\hline
		\textbf{anha} & \textbf{-ak} & \textbf{ifak} & \textbf{-k} & \textbf{dothrak} \\
		\hline
		\textbf{yer} & \textbf{-i} & \textbf{ifi} & \textbf{-e} & \textbf{dothrae} \\
		\hline
\end{tabular}
\section*{Exercises}
\begin{enumerate}
	\item Conjugate the following verbs in the negative present tense. You can assume here that
\textbf{-lat} verbs end in \textbf{-lat}.
	\begin{enumerate}
		\item \textbf{aranat}
		\item \textbf{arthasolat}
		\item \textbf{atholat}
		\item \textbf{emat}
	\end{enumerate}
\end{enumerate}
