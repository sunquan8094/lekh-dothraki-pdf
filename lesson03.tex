\chapter{Accusative Case of Inanimate Nouns}
\section*{Vocabulary}
\begin{itemize}
	\item \textbf{mai} [\textit{na}]: mother
	\item \textbf{yalli} [\textit{ni}]: child
	\item \textbf{vado} [\textit{ni}]: turnip
	\item \textbf{sewafikh} [\textit{ni}]: wine
	\item \textbf{dranekh} [\textit{ni}]: human milk
	\item \textbf{lamekh} [\textit{ni}]: mare's milk
	\item \textbf{adakhat} [\textit{v-at}]: to eat
	\item \textbf{indelat} [\textit{v-lat}]: to drink
\end{itemize}
\section*{Text}
	\textbf{Mai adakha eshin. Yalli adakha vad. Hrakkar adakha alegre.}
	\textbf{Mai indee sewafikh. Yalli indee dranekh. Hrakkar indee lamekh.}
\section*{Grammar}
\subsection*{Inanimate Nouns}
	Inanimate nouns are one of two types of nouns in Dothraki, the other being
  animate nouns. Inanimate nouns usually refer to the visibly lifeless and
	passive objects.
\subsection*{Noun Declension}
	In Dothraki, nouns are to be declined depending on its case.
	Only animate nouns decline by number as well, as we will see in a later lesson.
	The possible cases are nearly identical in function to those of Latin. The
	nouns you have learned in previous lessons were given to you in the nominative
	case, which is the case a noun assumes when it is the subject of a sentence.
 	In this lesson, we will be focusing on the accusative case, the case a noun
 	assumes when it is the direct object of a sentence.
\subsection*{Declining Inanimate Nouns in the Accusative Case}
	If the noun ends in a consonant, it is left as is. If the noun ends in a vowel,
	the final vowel is dropped. Note that epenthesis rules apply in this case.
\section*{Exercises}
