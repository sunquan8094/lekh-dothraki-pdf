\chapter*{Pronunciation}
\addcontentsline{toc}{chapter}{Pronunciation}
\section*{Vowels}
\begin{itemize}
	\item \textbf{a}: as in English \textit{father}
	\item \textbf{e}: as in English \textit{get}
	\item \textbf{i}: as in English \textit{sheesh}
	\item \textbf{o}: as in Japanese \textit{toki}
\end{itemize}
\section*{Consonants}
\begin{itemize}
	\item \textbf{b}:
	\item \textbf{ch}:
	\item \textbf{d}:
	\item \textbf{f}:
	\item \textbf{g}:
	\item \textbf{h}:
	\item \textbf{j}:
	\item \textbf{k}:
	\item \textbf{kh}:
	\item \textbf{l}:
	\item \textbf{m}:
	\item \textbf{n}:
	\item \textbf{p}:
	\item \textbf{q}:
	\item \textbf{r}:
	\item \textbf{s}:
	\item \textbf{sh}:
	\item \textbf{t}:
	\item \textbf{th}:
	\item \textbf{v}:
	\item \textbf{w}:
	\item \textbf{y}:
	\item \textbf{z}:
	\item \textbf{zh}:
\end{itemize}
\section*{Epenthesis}
Epenthesis is the addition of a sound to a word in order to make pronunciation
easier. Dothraki has a few rules regarding epenthesis:
\begin{itemize}
	\item Words must not end in \textbf{w}, \textbf{g}, or \textbf{q}. Add an extra \textbf{e} to the
				end should this occur.
	\item Words must not end in double (geminate) consonants.
        Add an extra \textbf{e} to the end should this occur.
	\item Given an arbitrary consonant \textbf{C}, words must not end in
				\textbf{Cw}, \textbf{Cr}, \textbf{Cl}, or \textbf{Cy}.
				Add an extra \textbf{e} to the end should this occur.
\end{itemize}
There are exceptions to this rule:
\begin{itemize}
	\item When nouns end with the pattern \textbf{CCV}. Such a noun is classified as
				irregular.
\end{itemize}
